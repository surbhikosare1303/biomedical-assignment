\documentclass[12pt]{article}

\usepackage{graphicx}
\graphicspath{{downloads/}}
\usepackage[utf8]{inputenc} % Required for inputting international characters
\usepackage[T1]{fontenc} % Output font encoding for international characters

\usepackage{mathpazo} % Palatino font




\begin{document}
\date{}
%----------------------------------------------------------------------------------------
%	TITLE PAGE
%----------------------------------------------------------------------------------------

\begin{titlepage} % Suppresses displaying the page number on the title page and the subsequent page counts as page 1
	\newcommand{\HRule}{\rule{\linewidth}{0.5mm}} % Defines a new command for horizontal lines, change thickness here
	
	\center % Centre everything on the page
	
	%------------------------------------------------
	%	Headings
	%------------------------------------------------
	
	\textsc{\LARGE NATIONAL INSTITUTE OF TECHNOLOGY   RAIPUR}\\[1.5cm] % Main heading such as the name of your university/college
	
	\textsc{\Large BIOMEDICAL ENGINEERING }\\[0.5cm] % Major heading such as course name
	
	\textsc{\large ASSIGMENT}\\[0.5cm] % Minor heading such as course title
	
	
	
	
	
	%------------------------------------------------
	%	Title
	%------------------------------------------------
	
	\HRule\\[0.4cm]
	
	{\huge\bfseries Medical Devices}\\[0.4cm] % Title of your document
	
	\HRule\\[1.5cm]
	
	%------------------------------------------------
	%	Submitted by;(s)
	%------------------------------------------------
	
	\begin{minipage}{0.4\textwidth}
		\begin{flushleft}
			\large
			\textit{submitted by}\\
			Name- \textsc{Surbhi Kosare} % 
			Roll no.-21111066 %
			
			
		\end{flushleft}
	\end{minipage}
	~
	\begin{minipage}{0.4\textwidth}
		\begin{flushright}
			\large
			\textit{Under the supervision of}\\
			Dr. Saurab \textsc{Gupta} % Supervisor's name
		\end{flushright}
	\end{minipage}
	
	% If you don't want a supervisor, uncomment the two lines below and comment the code above
	%{\large\textit{Author}}\\
	%John \textsc{Smith} % Your name
	
	%------------------------------------------------
	%	
	%------------------------------------------------
	
	\vfill\vfill\vfill % Position the date 3/4 down the remaining page
	
	{\large\today} % Date, change the \today to a set date if you want to be precise
	
	%------------------------------------------------
	%	Logo
	%------------------------------------------------
	
	%\vfill\vfill
	%\includegraphics[width=0.2\textwidth]{placeholder.jpg}\\[1cm] % Include a department/university logo - this will require the graphicx package
	 
	%----------------------------------------------------------------------------------------
	
	\vfill % Push the date up 1/4 of the remaining page
	\date{}
\end{titlepage}
	
\clearpage
\newpage







\section{UV Crosslinker}


\begin{figure}[h]
\centering
\includegraphics[scale=0.5]{uv}

\end{figure}





\subsection{Introduction}

Researchers can use the UV Crosslinker to expose samples to a controlled amount of ultraviolet radiation quickly, safely, and efficiently. Crosslinking of DNA or RNA to nitrocellulose, nylon, or reinforced nitrocellulose occurs when samples are exposed to UV.

The Crosslinkers are used to monitor and control the amount of ultraviolet (UV) radiation that enters the exposure chamber.

A revolutionary UV sensor continuously measures UV energy and responds to variations in UV intensity automatically.

As the UV bulbs get older, this happens. You can set UV levels using the same UV sensor feedback measuring system.

When the prescribed UV energy dose has been reached, the UV sources are automatically deactivated.

attained.


\subsection{Working}

The Crosslinker is a laboratory-based multi-purpose UV exposure device. In the laboratory, UV radiation has a wide range of applications.

\subsection{Applications}


 
\subsubsection{.}

A.After Northern, Southern, slot, or dot blotting, UV crosslinking of DNA and RNA is achieved by covalently binding nucleic acids to nitrocellulose.

\subsubsection{B.In AgarGelsose, nicking ethidium-bromide stained DNA} 

\subsubsection{C.Creating cleavage-inhibiting thymine dimers using gene mapping}


\subsubsection{E. Ultraviolet Sterilization}

\subsubsection{F. UV Curing}

\subsection{Stratagene System}

The UV Stratalinker 2400 crosslinker from Stratagene is used to crosslink DNA and RNA to nylon, nitrocellulose, and nylon-reinforced nitrocellulose membranes. In contrast to the usual procedure of baking filters at 80°C for 2 hours, the crosslinking process takes only 25–50 seconds. Furthermore, when compared to oven-baking, crosslinking has been demonstrated to dramatically boost hybridization signals. Each Stratalinker UV crosslinker is fitted with an inbuilt photodetector that compensates for the natural shift in power output of ageing ultraviolet lamps for maximum crosslinking performance. The Stratalinker UV crosslinker can be used for Northern, Southern, dot, or slot blot analysis, colony or plaque screening, nicking DNA in agarose gels before blotting, dimer formation to perform partial digests for rapid restriction mapping, and UV sensitivity testing for colony or plaque screening.

\subsection{Effects}

Collagen-fiber shrinkage temperature, resistance to degradation in collagenase, and durability under load were all improved by UV or DHT crosslinking.


\clearpage


\section{HPLC}

\subsection{Principle of HPLC}

The distribution of the analyte (sample) between a mobile phase (eluent) and a stationary phase is the basis for HPLC separation (packing material of the column). The molecules are slowed while passing through the stationary phase, depending on the chemical structure of the analyte. The time a sample spends "on-column" is determined by the specific intermolecular interactions between its molecules and the packing material. As a result, the eluted elements of a sample occur at different times. As a result, the sample ingredients can be separated.
After the analytes have left the column, a detection unit (such as a UV detector) detects them. A data management system (computer software) converts and records the signals, which are then displayed in a chromatogram. The mobile phase can be used after passing via the detecting unit.

\includegraphics[scale=0.4]{cap}


\subsection{Basic HPLC system components}



\subsubsection{Solvent Degasser} – as the solvents travel to the HPLC pump, it eliminates air gases.

\subsubsection{HPLC Pump} – allows for the flow of solvent and the proportioning of it.

\subsubsection{ Autosampler} – The pump collects samples from vials and injects them into the solvent flow.

\subsubsection{Detector} –responds to the separated analytes coming out of the HPLC column and generates a signal for the programme.

 \subsubsection{column oven}  – The HPLC column is housed here, and the temperature is maintained at a consistent level for repeatable separations.

\subsection{Gradient vs Isocratic}

Two alternative modalities are often applicable depending on the makeup of the mobile phase. An isocratic elution system is one in which the mobile phase composition remains constant throughout the separation process. The HPLC system is called a gradient elution system when the mobile phase composition changes during separation.There are two main approaches for using a gradient system: a low-pressure gradient (LPG) and a high-pressure gradient (HPG) (HPG). A low-pressure gradient indicates that the solvents are mixed upstream of the pump (suction side). The separate solvents are delivered by distinct pumps and blended after the pumps in a high-pressure gradient system (discharge side).

\clearpage

\section{Hysteroscope}

\subsection{Introduction}

An endoscope with optical and light tubes or fibres is known as a hysteroscope. It is inserted into a sheath that has an inflow and outflow channel for uterine cavity insufflation. An operating channel may also be present to allow scissors, graspers, or biopsy instruments to be introduced. [1] A hysteroscopic resectoscope is similar to a transurethral resectoscope in that it permits an electric loop to be inserted to shave off tissue, such as a fibroid.  A touch hysteroscope is one that does not require the use of distention material.

\subsection{Procedure}

In hospitals, surgical centres, and doctors, offices, hysteroscopy has been performed. It is best performed after a menstruation, when the endometrium is somewhat thin. On adequately selected patients, both diagnostic and simple surgical hysteroscopy can be performed in an office or clinic environment. It is possible to apply local anaesthetic. It is not always necessary to use analgesics. A Lidocaine injection in the upper section of the cervix can be used to induce a paracervical block. Under general anaesthesia (endotracheal or laryngeal mask) or Monitored Anesthesia Care, hysteroscopic intervention can be performed (MAC). Antibiotics are not required as a preventative measure. During the procedure, the patient is in a lithotomy position.


\subsection{Cervical dilation}

The contemporary hysteroscope has a tiny enough diameter to pass into the cervix without difficulty. Cervical dilatation may be required for a portion of women prior to insertion. Cervical dilation is accomplished by extending the cervix using a succession of dilators of varying diameters. Only in premenopausal women does misoprostol before hysteroscopy for cervical dilatation appear to make the process easier and less complex.

\subsection{Insertion and inspection}

The sheath of the hysteroscope is introduced transvaginally into the uterine cavity, the cavity is insufflated, and an examination is conducted.

\includegraphics[scale=0.3]{hystero}

\subsection{Complications}
Uterine perforation occurs when the hysteroscope or one of its operating tools punctures the uterine wall. This can result in blood loss and organ damage. Peritonitis can be fatal if other organs, such as the bowel, are affected during a perforation. Cervical laceration, intrauterine infection (particularly during lengthy treatments), electrical and laser damage, and difficulties induced by the distention media are all possible issues.
Because of embolism or fluid overload with electrolyte imbalances, the use of insufflation (also known as distending) media can result in serious and even deadly consequences.

\subparagraph{•} Because oestrogen levels are higher in women of reproductive age, they are more susceptible to develop hyponatremic encephalopathy.Using previous procedures, the total complication rate for diagnostic and surgical hysteroscopy was very less, with major problems occurring in less  of cases.  Morcellation has a lower rate of complications  than electrocautery.

\clearpage


\section{CPAP}

\includegraphics[scale=0.5]{cpap}


\subsection{Introduction}

In persons who are breathing spontaneously, continuous positive airway pressure (CPAP) is a type of positive airway pressure in which air is injected into the airways to maintain a constant pressure to keep the airways open. The pressure in the alveoli above atmospheric pressure at the end of expiration is known as positive end-expiratory pressure (PEEP). PEEP is delivered using CPAP, which also maintains the set pressure throughout the breathing cycle, including inspiration and expiration. [1] Water pressure is measured in centimetres (cm H2O). Bilevel positive airway pressure (BiPAP) differs from CPAP in that the pressure supplied varies depending on whether the patient is inhaling or exhaling. Inspiratory positive airway pressure (IPAP) and expiratory positive airway pressure (EPAP) are the terms for these pressures.


CPAP does not supply more pressure above the predetermined level, and patients must initiate all of their breaths.

The use of CPAP preserves PEEP, reduces atelectasis, increases the alveolus' surface area, improves V/Q matching, and so improves oxygenation. It can also help with ventilation indirectly, while CPAP alone is typically insufficient for non-invasive breathing, necessitating extra pressure support during inspiration (IPAP on BiPAP).

\subsection{Anatomy and Physiology}

Patients inhale air through their nose, which goes through the nasopharynx, oropharynx, larynx, trachea, bronchi, bronchioles, and alveoli. Excess tissue, tonsillar overgrowth, poor muscle tone, fatty excess, and secretions, among other things, can obstruct parts of the respiratory tract. CPAP helps to maintain the airways open and prevents collapse by delivering forceful air.

\subsection{Contadictions}

Individuals who can not breathe spontaneously cannot utilise CPAP. Invasive ventilation or non-invasive ventilation with CPAP plus additional pressure support and a backup rate are required for patients with low respiratory drive (BiPAP).


The following are relative contraindications for CPAP:

\subparagraph{•}
Patient who is uncooperative or has a high level of anxiety

\subparagraph{•}
Consciousness impairment and incapacity to protect their airway

\subparagraph{•}
Cardiorespiratory instability or respiratory arres

\subparagraph{•}
Surgery on the face, oesophagus, or stomach

\subparagraph{•}
Severe nausea with vomiting

\subsection{Enhancing healthcare team outcomes}

Patients with obstructive sleep apnea are frequently prescribed CPAP by their primary care provider, nurse practitioner, internist, or neurologist. Patient education, on the other hand, is critical for optimal compliance. Because of its pain, many patients only use these devices for a short time. The use of a continuous positive airway pressure (CPAP) machine is just a temporary solution for obstructive sleep apnea and does not reduce the risk of cardiac problems. Patients should be urged to lose weight, eat healthily, quit smoking, and exercise on a regular basis at the same time.

\section{Tonometer}

\subsection{Introduction}

The non-pigmented ciliary epithelium produces aqueous fluid in the eye, which nourishes the cornea, lens, and trabecular meshwork. The eye is pressurised by the balance between its production and drainage, which is quantified as intraocular pressure (IOP). Glaucoma is a leading cause of irreversible blindness globally, and IOP is a key factor in its development and progression. IOP reduction has been demonstrated to postpone or prevent the formation of glaucoma in people with high IOP and to reduce the course of glaucoma in people who already have it [4, 5]. IOP measurement and management are critical components of glaucoma monitoring and treatment.

Tonometry is an important part of an ophthalmologist's routine checkup. This study will go through the most prevalent types of tonometers on the market today, as well as their applicability, benefits, and drawbacks.



\includegraphics[scale=0.13]{tono}


\subsection{Applanation tonometry}

Applanation tonometry is based on the Imbert–Fick rule, which states that the force necessary to flatten a specific section of an infinitely thin and flexible membrane can be used to determine the pressure inside the sphere. Several tools flatten a region of the cornea by applying an applanating force.

\subsection{Conclusion}

IOP measurement that is accurate and precise is a key part of glaucoma management. Tonometers come in a variety of shapes and sizes. The GAT has the benefit of a lengthy history of use, as well as reliability and repeatability. Furthermore, GAT has been used as the gold standard in almost all glaucoma clinical trials. Infections and corneal abrasions are less likely with non-contact devices such as the NCT, ORA, and Corvis ST. Tonometers like the Icare, which are portable and don't require anaesthesia or specialist training to use, can be beneficial for screening and in the clinic. In circumstances where patients are unable to sit straight, devices such as the Tono-Pen and Perkins can be useful.

Corneal characteristics may be more impacted by devices that generate fast deformation of the cornea, such as the NCT. Newer devices, such as the ORA and DCT, are said to be less affected by corneal characteristics, although they aren't extensively utilised right now. In summary, different types of tonometers differ in terms of portability, placement, agreement with the GAT, corneal characteristics' influence, and the necessity for qualified operators and patient cooperation. When choosing the right instrument for a certain purpose, setting, and patient population, these aspects should be taken into account.

\end{document}
