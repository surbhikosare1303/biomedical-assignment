\documentclass[12pt]{article}


\usepackage[utf8]{inputenc}
\usepackage{mathpazo}
\usepackage[T1]{fontenc}

\usepackage{graphicx}
\graphicspath{{images/}}
\usepackage[labelfont=bf]{caption}
\begin{document}
\begin{titlepage}
\newcommand{\HRule}{\rule{\linewidth}{0.5mm}}
\center
\textsc{\LARGE National Institute of Technology Raipur}\\[1.0cm]

\textsc{\Large Biomedical Engineering}\\[0.5cm] 
\textsc{\large Assignment}\\[0.5cm]
\HRule\\[0.4cm]
	
	{\huge\bfseries Future of Healthcare}\\[0.1cm]
	\HRule\\[1.1cm]
	\begin{minipage}{0.4\textwidth}
		\begin{flushleft}
			\large
			\textit{Submitted By:}\\
          		Name : Surbhi Kosare \\
			   Roll No. : 21111066\\
			   Semester : First\\
			   Branch - Biomedical Engineering 
			   
			\end{flushleft}
	\end{minipage}
	~
	\begin{minipage}{0.5\textwidth}
		\begin{flushright}
			\large
			\textit{Under The Supervision Of:}\\
			Dr. Saurabh Gupta\\
			Department Of Biomedical Engineering\\
			NIT Raipur
		\end{flushright}
	\end{minipage}
	\vfill\vfill\vfill 
	
	
	\vfill 
\end{titlepage}

\section{Future Of Healthcare}

\subsection{What is the Future of Healthcare}

We are only around 20 years away from the health future we imagine, yet health in 2040 will be a world apart from what we have now. We can reasonably expect digital transformation—enabled by radically interoperable data, artificial intelligence (AI), and open, secure platforms—to drive much of this change, based on emerging technology. We believe that, unlike today, care will be organised around the consumer rather than the institutions that currently drive our health-care system.


Streams of health data, together with data from a range of other relevant sources, will merge by 2040 (and possibly much before) to generate a complex and highly individualised picture of each consumer's health. Wearable devices that track our steps, sleep habits, and even heart rate have become an integral part of our lives in ways we could never have imagined only a few years ago. This is a pattern we believe will continue. For example, the next generation of sensors will transition us away from wearable gadgets and toward invisible, always-on sensors implanted in the objects that surround us.

Consumers will almost certainly demand that their health information be portable once they have this extremely detailed personal information about their own health. Consumers have been accustomed to transformations in other industries, such as e-commerce and mobility. These customers will demand that health follow suit and become a seamless part of their lives, and they will vote with their feet and wallets.

\subsection{Impacts}

Existing stakeholders, newcomers, employers, and consumers will all be affected by the future of health. Many incumbents are naturally wary about bringing change to a market that they already control. These firms may be well-positioned to lead from the front, given their solid foundation in the existing ecosystem and ability to manage the regulatory framework.

Google, Amazon, and Apple9, for example, are disrupting and reshaping the market. Legacy stakeholders should think about whether they should disrupt themselves or isolate and protect their services in order to keep part of their current market share. Some incumbent businesses may succumb to competition from beyond the established industry boundaries, whereas others may be able to help usher in the future of health.
By 2040, we expect successful businesses to identify and compete in one or more of the new company archetypes depicted in Figure 4, taking into account their current capabilities, primary missions and beliefs, and future expectations.

\subsection{AI in healthcare}

The use of machine-learning algorithms and software, or artificial intelligence (AI), to replicate human cognition in the analysis, display, and comprehension of complicated medical and health-care data is referred to as artificial intelligence in healthcare. AI is defined as the ability of computer algorithms to make educated guesses based purely on input data.

\subsection{Conclusion}

In the United States, public health faces a conundrum. On the one hand, the advancements in health care for which public health was formed in this country are generally taken for granted: safe drinking water, significant protection against previously pandemic diseases, and an infant mortality rate that is one-tenth that of 1900. It's tough to keep a sense of urgency about these issues, but continued vigilance is required to protect the advances that have already been made.For example, our country's success in lowering newborn mortality has stalled: during the 1970s, infant mortality fell at a rate of 5 to 6 percent per year on average, but from 1981 to 1984, the pace of fall reduced to about 3 percent. Hughes et al. (Hughes et al., 1986) Infant mortality has recently surged in Detroit, Los Angeles, and other cities, and it remains alarmingly high in low-income areas. Measles outbreaks, for which an effective vaccine is available, continue to occur. Syphilis is on the rise once more. (Bureau of the Census, United States Department of Commerce, 1986) However, public health authorities' warnings concerning these incidents are sometimes viewed as self-serving.
\end{document}