\documentclass[12pt]{article}


\usepackage[utf8]{inputenc}
\usepackage{mathpazo}
\usepackage[T1]{fontenc}

\usepackage{graphicx}
\graphicspath{{images/}}
\usepackage[labelfont=bf]{caption}
\begin{document}
\begin{titlepage}
\newcommand{\HRule}{\rule{\linewidth}{0.5mm}}
\center
\textsc{\LARGE National Institute of Technology Raipur}\\[1.0cm]

\textsc{\Large Biomedical Engineering}\\[0.5cm] 
\textsc{\large Assignment}\\[0.5cm]
\HRule\\[0.4cm]
	
	{\huge\bfseries Solution for covid 19 by biomedical Engineer}\\[0.1cm]
	\HRule\\[1.1cm]
	\begin{minipage}{0.4\textwidth}
		\begin{flushleft}
			\large
			\textit{Submitted By:}\\
          		Name : Surbhi Kosare \\
			   Roll No. : 21111066\\
			   Semester : First\\
			   Branch - Biomedical Engineering 
			   
			\end{flushleft}
	\end{minipage}
	~
	\begin{minipage}{0.5\textwidth}
		\begin{flushright}
			\large
			\textit{Under The Supervision Of:}\\
			Dr. Saurabh Gupta\\
			Department Of Biomedical Engineering\\
			NIT Raipur
		\end{flushright}
	\end{minipage}
	\vfill\vfill\vfill 
	
	
	\vfill 
\end{titlepage}

\section{Introduction}
\subsection{Covid 19}
The SARS-COV 2 Virus causes COVID 19. Coronaviruses are a group of viruses that can cause anything from a typical cold to more serious illnesses. A novel coronavirus is a strain that has never been seen before.
\subsection{Role of Biomedical Engineer}
We've witnessed the critical role that medical technology plays in patient care with the tremendous surge in demand for ventilators and other equipment.

Many engineers have adapted their existing expertise and equipment to assist in the fight against COVID-19 as a result of the ongoing problem.
\section{Solutions by Biomedical Engineers}
\subsection{Manufacturing medical equipment}
The U.S. Army, Harvard, University of South Florida, Desktop Metal, Formlabs, and HP, among others, are collaborating with the Center for Design and Manufacturing Excellence (CDME) and Institute for Materials Research (IMR) to develop, test, and implement a strategy to produce COVID-19 test swabs, which are running dangerously low in supply.

CDME and IMR are working on the design and testing of the swabs, as well as spearheading the search for local manufacturers who are certified to make medical grade swabs to meet the demands of the Wexner Medical Center and the surrounding community.

\subsection{Pivoting to vaccines}
Professors of Biomedical Engineering Daniel Gallego-Perez and Natalia Higuita-Castro are working on innovative, safe, and targeted techniques for delivering mRNA vaccines in their labs. While the researchers' usual areas of interest are cancer and regenerative medicine, their technologies could be used to produce and distribute vaccines.

"We're working on ways to deliver mRNA vaccines to specific cell types in a highly targeted way," Gallego-Perez explained. "Once we've determined the best technique, we'll consult COVID-19 experts to see if we can move this study into the correct arena for future efficacy studies."
\subsection{Rapid response to PPE kit}
Face shields protect the face from fluids, spray, and droplets while increasing the life of N95 face masks, according to infectious disease experts.


In addition to 3D printing, an interdisciplinary team created injection moulded visors by April 6. On April 13, a first shipment of 312 fully constructed face shields was delivered to the Wexner Medical Center.
\subsection{Testing Innovation}
Professor Perena Gouma's team is working on a low-cost COVID-19 breathalyser that will test two essential infection biomarkers from a single exhaled breath. Prior to her arrival at Ohio State, she invented a hand-held breath monitor that could identify flu signs before symptoms manifest.


The non-invasive diagnostic tool, according to Gouma, will not only allow for early diagnosis of the illness, but also provide information on the disease's progression and severity. She is currently in discussions with government officials as well as business leaders.
\subsection{From epidemic to pandemic}
Professors Raghu Machiraju and Anish Arora of Computer Science and Engineering are using an NSF-funded project titled "BD Spoke: Community-Driven Data Engineering for Opioid and Substance Abuse in the Rural Midwest" to focus on the COVID-19 pandemic with College of Public Health Assistant Professor Ayaz Hyder. The project's overall purpose is to build data platforms and ecosystems to combat epidemics that are spreading. The OpenCOVID data commons will house datasets and tools that are particularly useful to local and regional public health organisations. National Children's Hospital, the University of Chicago, and the University of South Carolina are among the project's collaborators.

\end{document}