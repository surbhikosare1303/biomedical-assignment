\documentclass[12pt]{article}


\usepackage[utf8]{inputenc}
\usepackage{mathpazo}
\usepackage[T1]{fontenc}

\usepackage{graphicx}
\graphicspath{{images/}}
\usepackage[labelfont=bf]{caption}
\begin{document}
\begin{titlepage}
\newcommand{\HRule}{\rule{\linewidth}{0.5mm}}
\center
\textsc{\LARGE National Institute of Technology Raipur}\\[1.0cm]


\textsc{\Large Biomedical Engineering}\\[0.5cm] 
\textsc{\large Assignment}\\[0.5cm]
\HRule\\[0.4cm]
	
	{\huge\bfseries Evolution of Modern Health Care System}\\[0.1cm]
	\HRule\\[1.1cm]
	\begin{minipage}{0.4\textwidth}
		\begin{flushleft}
			\large
				
		\textit{Submitted By:}\\
          		Name : Surbhi Kosare \\
			   Roll No. : 21111066\\
			   Semester : First\\
			   Branch - Biomedical Engineering 
			   
			\end{flushleft}
	\end{minipage}
	~
	\begin{minipage}{0.5\textwidth}
		\begin{flushright}
			\large
			\textit{Under The Supervision Of:}\\
			Dr. Saurabh Gupta\\
			Department Of Biomedical Engineering\\
			NIT Raipur
		\end{flushright}
	\end{minipage}
	\vfill\vfill\vfill 
	
	
	\vfill 
\end{titlepage}

\section{Evolution of Modern Health Care System }

\subsection{Introduction}


New, better, and more effective solutions to problems are referred to as "innovation." The term has been applied to strategies, systems, technologies, ideas, services, and products that provide solutions to current healthcare issues. It originated in the business, technology, and marketing fields. The term "innovation" has become a buzzword in the healthcare business, with so many dynamic ways and approaches to select from. A clear, agreed-upon definition of the term has been missing from discussions regarding innovation. Because a lack of consensus is a barrier to introducing innovation into clinical practise, a precise definition is necessary. Because of a lack of definition and uniformity, the term "innovation" has been misused to characterise numerous advances in healthcare.

\subsection{STAKEHOLDER CONSIDERATIONS AND BARRIERS TO UPTAKE INNOVATIONS: ADOPTION AND IMPLEMENTATION}

According to Länsisalmi et al.10, the three components of innovation are I novelty, ii) application component, and iii) desired benefit. Although stakeholder factors must be considered, a 'intended benefit' should be centred on the recipient of care, the patient. Stakeholder issues are especially crucial when it comes to innovation adaptation and uptake. 10 With these elements in mind, the 'innovation process' can be better understood by analysing stakeholder groups' requirements, wants, and expectations. Other stakeholders to consider include physicians and other care providers, organisations, innovation businesses, and regulatory authorities, with patients at the forefront. When health innovation is successful, three major aspects are addressed: i) the way the patient is perceived, ii) the way the patient is heard, iii)the way patient is treated.


Even if the conditions are met, there are still impediments to the acknowledgment and adoption of healthcare innovations. Because diffusion is a social and dynamic process, it necessitates collaboration, communication, and knowledge exchange among those engaged. 11 As a result, adoption and implementation in healthcare entails a number of people, restrictions, and aspects that are unique to a system's social, political, policy, economic, institutional, and cultural context. 3,8,12,13 According to the Harvard Business Review, healthcare innovation may be broken down into three categories: I customer focus, ii) technology, and iii) business models. Stakeholders and their interests, finance and cost, policy and government rules, competition, and other changes are all elements that influence acceptance and diffusion in healthcare.

Each of these characteristics has an impact on whether something is deemed an innovation, as well as whether it is accepted and embraced in the healthcare sector. To put it another way, uptake necessitates stakeholders seeing a comparative benefit in adopting and executing the innovation. However, relative advantage does not guarantee adoption and implementation on its own. 5 Capacity, compatibility, complexity, trialability, observability, reinvention, and risk are all factors to consider. Stakeholders are more likely to adopt an innovation if it is congruent with their interests, simple to adopt, can be tested on a small scale, is observable, can be tweaked to fit their needs, and has minimum risk.

\subsection{Important considerations for the future of healthcare}

Our time travellers from a century ago were shocked by a few things, but with a few small bumps like William Harvey finding blood circulation, not little has changed all the way back to Hippocrates.


There are a lot of futures to consider. There will be another future as soon as we arrive to ours, and we will progressively see half-finished solutions displaced by much better ones. We may believe that all we need to do now is computerise all patient records, but before we're done, some flashy new technology will change what we want to do and how we should do it. We will have to live with fragmented and partially functional technologies for the foreseeable future.

We must take the future seriously since it is literally all we have, and it will undoubtedly be all our children's - and we can be assured that as we grow older, we will face all of the issues that come with ageing. Surely, we want the future of healthcare to be better? Not once, but on a regular basis, we should put work into future planning.
\subsection{Conclusion}
Making a cange plan is the first step in resolving a problem. When preparing for change in healthcare, it's common to hope that it will lead to an improvement or a solution to a problem that already exists. In reality, hardly every adjustment leads to a solution or improvement, let alone a breakthrough. Change may provide little to no gain or benefit, and in certain situations, it may even produce unintended negative repercussions. As a result, bringing about a change, no matter how big or tiny, cannot be called inherently "creative."

One of the keys to advances and developments in healthcare is observing the effects of change, whether it ends in failure or success. The requirements for health innovation have been met when the change is something new, or involves the process of introducing something new, and results in a benefit of improvement in the field of healthcare.









\end{document}